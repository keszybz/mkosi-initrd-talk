\documentclass[]{beamer}
%\usepackage[utopia]{mathdesign}
%\usepackage[no-math]{fontspec}
%\setmainfont{Liberation Serif}

\usepackage{minted}
% \usepackage{redhat}
\usepackage{hyperref}
\usepackage{ccicons}
\usepackage{ulem}

\usepackage{tikz}
\usetikzlibrary{arrows,shapes,snakes,automata,backgrounds,petri}

%gets rid of bottom navigation bars
\setbeamertemplate{footline}[frame number]{}

%gets rid of navigation symbols
\setbeamertemplate{navigation symbols}{}

\newcommand{\done}{\textcolor{teal}{\checkmark}}
\newcommand{\pull}[3]{\href{https://github.com/systemd/#1/pulls/#2}{#3 (\##2)}}
\newcommand{\pulldone}[3]{\pull{#1}{#2}{#3} \done}
\newcommand{\commit}[3]{\href{https://github.com/systemd/#1/commit/#2}{#3 (#2)}}
\newcommand{\commitdone}[3]{\commit{#1}{#2}{#3} \done}
\newcommand\pp\pause
%\newcommand\pp{}


\title{\textsc{Building initrds in a new way}}
\author{Zbigniew Jędrzejewski-Szmek}
\institute{%
  \includegraphics[width=0.4\textwidth]{images/Logo-redhat-color-375.png}\\
  \medskip
  \textit{zbyszek@in.waw.pl}\\
  \medskip
  \ccbysa
}
\date{\tiny FOSDEM 2023, 4.2.2023}

\begin{document}

\setbeamertemplate{itemize items}[square]

\begin{frame}
\titlepage % Print the title page as the first slide
\end{frame}

\begin{frame}
  \frametitle{(instead of) Intro}

  \pp

  see Lennart's \\
  \textsc{
    ``Image-Based Linux and TPMs\\
    \textit{Measured Boot, Protecting Secrets and you}''}

  \begin{itemize}
  \item SecureBoot signing used to protect code

  \item ``PCR measurements'', ``boot phases''

  \item
    systemd's «credentials» are a mechanism to pass identity information, certificates, key material, passwords, and~similar

  \item ``credentials'' locked to PCR state

  \item
    systemd's «system extensions» are a mechanism to dynamically extend the root fs

  \end{itemize}
\end{frame}


\begin{frame}
  \frametitle{Current approach to initrds}

%%   \pp
%%   goals:

%%   \begin{itemize}
%%   \item speed → event-driven logic
%%   \item speed → size minimization
%%   \item flexibility, versability, end-user choice
%%   \item local configuration embedded in the initrd
%%   \end{itemize}

%%   \pp
%%   results:

  \pp

  \begin{itemize}
  \item local builds pulling in files from host fs,\\
    \texttt{ldd} to resolve dependencies
  \pp

  \item the packaging layer is duplicated
  \pp

  \item lots of CPU cycles burnt during each kernel update
  \pp

  \quad

  \item at runtime: custom logic (e.g. dracut's initqueue)
  \pp

  \item custom tools (e.g. scripts to bring up LVM, dracut modules)
  \pp

  \item different execution environment
  \pp

  \item complexity (in particular when dracut is used with systemd)
  \pp

  \quad

  \item very little sharing of initrd logic between distros
  \end{itemize}
\end{frame}

\begin{frame}
  \frametitle{Consequences of signing and measurement}

  \pp
  \begin{itemize}
  \item signing of the kernel but not the initrd is a waste of CPU cycles \pause\\
  \item end-users want kernel+initrd signed by the distro \pause\\
  \item the initrd must be built by the distro \pause\\
  \item flexibility \& ability to inject local modifications — not useful \pause\\

    \quad

  \item if we are building in a package builder, let's build directly from distro packages\\
    (we \textit{could} build from files in the fs, but why?)
  \end{itemize}
\end{frame}

\begin{frame}
  \frametitle{Consequences of centralized builds}

  If we use pre-built images, two ways to deliver differentiated code
  \pause

  \begin{enumerate}
  \item initrd variants\pause
  \item systemd-sysexts
  \end{enumerate}
\end{frame}

%% \begin{frame}
%%   \frametitle{What are System Extensions good for?}

%%   \pp
%%   \textcolor{teal}{Reminder:}{ }\begin{minipage}[t]{0.8\linewidth}
%%     the initrd is a compressed cpio archive

%%     \quad

%%     a sysext is a GPT image with
%%     three partitions:\\
%%     the filesystem (e.g.\ compressed squashfs),
%%     dm-verity for the filesystem,
%%     signature for the verity data
%%   \end{minipage}

%%   \quad

%%   \begin{itemize}
%%     \pp
%%   \item a network configuration daemon + sshd

%%     \pp
%%   \item nfs / RAIDs / clevis / network storage

%%     \pp
%%   \item the full graphical stack\pp: a11y! \pp i18n!

%%     \pp
%%   \item the full sound stack\pp: a11y!

%%     \pp
%%   \item hardware enablement, incl. bluetooth

%%     \pp
%%   \item (suggestions welcome)
%%   \end{itemize}
%% \end{frame}

\begin{frame}
  \frametitle{Consequences of centralized builds, ctd.}

  If we use UKIs,
  we need a mechanism to replace config in the initrd
  \pause

  \begin{enumerate}
  \item automatic discovery (Discoverable Partitions Spec)\pause
  \item credentials for configuration\pause
  \end{enumerate}

  \vfill{}

  Easy building of sysexts depends build reproducibility of the initrd
\end{frame}


%% \begin{frame}
%%   \frametitle{What does the kernel say?}

%%   \pp
%%   (the short answer: it doesn't care)

%%   \hfill

%%   \pp
%%   the long answer:
%%   the initrd is just an in-memory file system\\
%%   \texttt{/init} is started instead of \texttt{/sbin/init}
%% \end{frame}

%% \begin{frame}
%%   \frametitle{New goals}

%%   \begin{itemize}
%%     \pp
%%   \item reuse distro packaging
%%     \pp
%%   \item use systemd in the initrd
%%     \pp
%%   \item use normal services
%%     \pp
%%   \item standard userspace environment
%%     \pp
%%   \item reasonable size
%%     \pp

%%   \quad

%%   \item build reproducible initrd images
%%     \pp
%%   \item build initrd images on vendor systems
%%     \pp
%%   \item sign the kernel + initrd
%%     \pp
%%   \item build a set of System Extension images for the initrd
%%     \pp
%%   \item sign those too
%%     \pp

%%   \quad

%%   \item maintainers of user-space packages handle ``initrd bugs''
%%   \end{itemize}
%% \end{frame}

\begin{frame}[fragile]
  \frametitle{mkosi — introduction}

  \url{https://github.com/systemd/mkosi}

  \pp

  \begin{itemize}
  \item A program that builds ``images'' from packages (and sources)
     \pp
   \item Support for GPT, verity, and signatures → sysexts
     \pp
   \item Also archives (\texttt{cpio}) → initrd/initramfs
%%     \pp
%%   \item Uses dnf / apt / pacman / zypper / … as backend\\
%%     \pp
%%     $\longrightarrow{}$ supports Fedora / CentOS / RHEL / Mageia / OpenMandriva / Debian / Ubuntu / Arch / OpenSUSE / Gentoo (sic!)
%%     \pp
%%   \item \texttt{mkosi.skeleton/}, \texttt{mkosi.extra/}, \\
%%         \texttt{mkosi.build}, \texttt{mkosi.postinst}, \texttt{mkosi.finalize}
%%     \pp
%%   \item Support for incremental builds
%%     \pp
%%   \item Writes ``manifests'' of installed packages
%%   \item We are working on build reproducibility
  \end{itemize}
\end{frame}

\begin{frame}[fragile]
  \frametitle{mkosi-initrd}

  \url{https://github.com/systemd/mkosi-initrd}

  \quad

  \pp

  \begin{itemize}
  \item mostly a series of a config files for \texttt{mkosi}
    \pp
  \item list of packages for the ``basic'' initrd
    \pp
  \item \texttt{mkosi.finalize} to set \texttt{/etc/initrd-release}
    \pp
  \item set of configs for sysexts

  \quad

    \pp
  \item also \texttt{/usr/lib/kernel/install.d/50-mkosi-initrd.install}

  \quad

    \pp
  \item hopefully coming soon to a Fedora install near you\\
    \small{\url{https://fedoraproject.org/wiki/Changes/mkosi-initrd}}
    
%%   \quad
%%     \pp
%%   \item ``systemd credentials'' will be used for configuration and local assets
  \end{itemize}
\end{frame}

\begin{frame}
  \frametitle{Results}

  \begin{itemize}
  \pp
  \item We get fully functional initrds.

  \pp
  \item The initrds are \textbf{bigger}.\\
  \pp
  Most of the difference is caused by kernel modules.

  \pp
  \item Only some subset of installations is supported.
  \end{itemize}
\end{frame}


\begin{frame}
  \frametitle{Benefits}

  \begin{itemize}
  \item \textbf{less} things
    \pp
  \item we use package dependency resolution mechanism
    \pp
  \item we let rpm/deb/pacman handle 90\% of the installation
    \pp
  \item we don't pull files from the host
    \pp
  \item images can be reproducible
    \pp
  \item images are the same for everyone
    \pp
  \item images can be easily signed
    \pp
  \item systemd does the heavy lifting in the initrd
%%     \pp
%%   \item bash helpers → compiled programs
%%     \pp
%%   \item developers don't need to learn another system\\
%%     (initrd is like a normal system,\\
%%     \phantom( just on an fs not backed by a disk)
%%     \pp
%%   \item clear ownership of bugs
%%     \pp
%%   \item initrd infrastructure can be shared between distros
  \end{itemize}
\end{frame}

\begin{frame}
  \frametitle{Objections?}

% If you have some experience in this area,
% and you think that this just infeasible/inefficient/unreasonable, consider that:

  \begin{itemize}
    \pp
    \item 
      systemd was already used in the initrd
    \pp
    \item 
      the first thing systemd does is to set up the environment
    \pp
    \item 
      having tools that support running in a custom environment is hence not useful
    \pp
    \item 
      after removing custom logic we don't need to add anything back
    \pp
    \item 
      the ecosystem is moving away from scripts towards compiled daemons
    \pp
    \item 
      most of the code is in shared libraries, which are installed in full because of link dependencies
    \pp
    \item 
      error handling, timeouts, retries, localized messages, event-driven logic, netlink,
      D-bus, all are much easier with ``real'' code
  \end{itemize}
\end{frame}


\begin{frame}
  \frametitle{Progress over the last few months}
  
  \begin{itemize}
  \item \texttt{mkosi-initrd} has a growing test suite (booting different storage types)
  \item \texttt{mkosi} supports builds as an unprivileged user
  \item \texttt{systemd} is getting new credential features
  \item \texttt{systemd} has new \texttt{ukify} helper to builds UKIs
  \item Fedora 38 Change accepted for Unified Kernel Images for VMs
  \item New kernel package split in Fedora (\texttt{kernel-modules-core} finally)
  \item GRUB2 might get support for UKIs (\url{https://github.com/osteffenrh/grub2})
  \item Fedora 39 Change proposal for \texttt{mkosi-initrd}
  \end{itemize}
\end{frame}
  


%% \begin{frame}
%%   \frametitle{Boot loader integration}

%%   Credentials and sysexts are loaded by systemd-stub.\\
%%   Loading \texttt{should} work with any boot loader.

%%   % \quad
%% \end{frame}

\begin{frame}[fragile]
  \frametitle{Links}

  \url{https://github.com/systemd/mkosi}

  \url{https://github.com/systemd/mkosi-initrd}

  \url{https://www.freedesktop.org/software/systemd/man/systemd-sysext.html}

  {
    \small
    \url{https://gitlab.com/cryptsetup/cryptsetup/-/wikis/DMVerity}\\
    \color{gray}{\url{https://www.kernel.org/doc/html/latest/admin-guide/device-mapper/verity.html}}
    }

  \url{https://www.kernel.org/doc/html/latest/filesystems/overlayfs.html}

  \quad

  These slides:
  \url{https://github.com/keszybz/mkosi-initrd-talk/raw/main/fosdem2023-buildling-initrds-in-a-new-way.pdf}

  \quad
  \pp

  \hfill \textcolor{red}{QUESTIONS?} \textcolor{green}{\bf /} \textcolor{red}{EOF} \hfill{}

\end{frame}


\end{document}
