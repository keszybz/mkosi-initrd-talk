\documentclass[]{beamer}
%\usepackage[utopia]{mathdesign}
%\usepackage[no-math]{fontspec}
%\setmainfont{Liberation Serif}

\usepackage{minted}
% \usepackage{redhat}
\usepackage{hyperref}
\usepackage{ccicons}
\usepackage{ulem}

\usepackage{tikz}
\usetikzlibrary{arrows,shapes,snakes,automata,backgrounds,petri}

%gets rid of bottom navigation bars
\setbeamertemplate{footline}[frame number]{}

%gets rid of navigation symbols
\setbeamertemplate{navigation symbols}{}

\newcommand{\done}{\textcolor{teal}{\checkmark}}
\newcommand{\pull}[3]{\href{https://github.com/systemd/#1/pulls/#2}{#3 (\##2)}}
\newcommand{\pulldone}[3]{\pull{#1}{#2}{#3} \done}
\newcommand{\commit}[3]{\href{https://github.com/systemd/#1/commit/#2}{#3 (#2)}}
\newcommand{\commitdone}[3]{\commit{#1}{#2}{#3} \done}
% \newcommand\pp\pause
\newcommand\pp{}


\title{\textsc{New design for initrds}}
\author{Zbigniew Jędrzejewski-Szmek}
\institute{%
  \includegraphics[width=0.4\textwidth]{images/Logo-redhat-color-375.png}\\
  \medskip
  \textit{zbyszek@in.waw.pl}\\
  \medskip
  \ccbysa
}
\date{\tiny LPC 2022, 12.09.2022}

\begin{document}

\setbeamertemplate{itemize items}[square]

\begin{frame}
\titlepage % Print the title page as the first slide
\end{frame}

\begin{frame}
  \frametitle{(instead of) Intro}

  \pp

  see Lennart's \\
  \textsc{
    ``Towards Secure Unified Kernel Images\\
    for Generic Linux Distributions and Everyone Else''}

  \pp

  \begin{itemize}
  \item ``let's pre-build initrds in vendor build system''

    \pp
  \item ``System extension images are GPT disk images, implementing the Discoverable Partitions Specification''

    \pp
  \item ``Signed as one for SecureBoot''

    \pp
  \item
    ``systemd's «service credentials» are a concept for passing identity information, certificates, key material, passwords, and~similar to services''

    \quad

    \pp
  \item also: reproducible builds

  \pp
  \item also: a simpler system

  \end{itemize}
\end{frame}

\begin{frame}
  \frametitle{Current approach to initrds}

  \pp
  goals:

  \begin{itemize}
  \item speed → event-driven logic
  \item speed → size minimization
  \item flexibility, versability, end-user choice
  \item local configuration embedded in the initrd
  \end{itemize}

  \pp
  results:

  \begin{itemize}
  \item local builds
  \pp

  \item custom logic (e.g. dracut's initqueue)
  \pp

  \item custom tools (e.g. scripts to bring up lvm, dracut modules)
  \pp

  \item a unique execution environment
  \pp

  \item the packaging layer is duplicated
  \pp

  \item complexity (in particular when dracut is used with systemd)
  \pp

  \item lots of CPU cycles burnt during each kernel update
  \end{itemize}
\end{frame}

\begin{frame}
  \frametitle{What does the kernel say?}

  \pp
  (the short answer: it doesn't care)

  \hfill

  \pp
  the long answer:
  the initrd is just an in-memory file system\\
  \texttt{/init} is started instead of \texttt{/sbin/init}
\end{frame}

\begin{frame}
  \frametitle{New goals}

  \begin{itemize}
    \pp
  \item reuse distro packaging
    \pp
  \item use systemd in the initrd
    \pp
  \item use normal services
    \pp
  \item standard userspace environment
    \pp
  \item reasonable size
    \pp

  \quad

  \item build reproducible initrd images
    \pp
  \item build initrd images on vendor systems
    \pp
  \item sign the kernel + initrd
    \pp
  \item build a set of System Extension images for the initrd
    \pp
  \item sign those too
    \pp

  \quad

  \item maintainers of user-space packages handle ``initrd bugs''
  \end{itemize}
\end{frame}

\begin{frame}
  \frametitle{What are System Extensions good for?}

  \pp
  \textcolor{teal}{Reminder:}{ }\begin{minipage}[t]{0.8\linewidth}
    the initrd is a compressed cpio archive

    \quad

    a sysext is a GPT image with
    three partitions:\\
    the filesystem (e.g.\ compressed squashfs),
    dm-verity for the filesystem,
    signature for the verity data
  \end{minipage}

  \quad

  \begin{itemize}
    \pp
  \item a network configuration daemon + sshd

    \pp
  \item iSCSI / nfs / RAIDs / clevis / storage

    \pp
  \item the full graphical stack\pp: a11y! \pp i18n!

    \pp
  \item the full sound stack\pp: a11y!

    \pp
  \item hardware enablement, incl. bluetooth

    \pp
  \item (suggestions welcome)
  \end{itemize}
\end{frame}

\begin{frame}[fragile]
  \frametitle{Digression: mkosi}

  \url{https://github.com/systemd/mkosi}

  \pp

  \begin{itemize}
  \item A program that builds ``images'' from packages (and sources)
    \pp
  \item Support for GPT, verity, and signature
    \pp
  \item Also \texttt{cpio} images (initrd/initramfs)
    \pp
  \item Uses dnf / apt / pacman / zypper / … as backend\\
    \pp
    $\longrightarrow{}$ supports Fedora / CentOS / RHEL / Mageia / OpenMandriva / Debian / Ubuntu / Arch / OpenSUSE / Gentoo (sic!)
    \pp
  \item \texttt{mkosi.skeleton/}, \texttt{mkosi.extra/}, \\
        \texttt{mkosi.build}, \texttt{mkosi.postinst}, \texttt{mkosi.finalize}
    \pp
  \item Support for incremental builds
    \pp
  \item Writes ``manifests'' of installed packages
  \item We are working on build reproducibility
  \end{itemize}
\end{frame}

\begin{frame}[fragile]
  \frametitle{mkosi-initrd}

  \url{https://github.com/systemd/mkosi-initrd}

  \quad

  \pp

  \begin{itemize}
  \item mostly a series of a config files for \texttt{mkosi}
    \pp
  \item list of packages for the ``basic'' initrd
    \pp
  \item \texttt{mkosi.finalize} to set \texttt{/etc/initrd-release}
    \pp
  \item set of configs for sysexts

  \quad

    \pp
  \item also \texttt{/usr/lib/kernel/install.d/50-mkosi-initrd.install}

  \quad
    \pp
  \item ``systemd credentials'' will be used for configuration and local assets
  \end{itemize}
\end{frame}

\begin{frame}
  \frametitle{Benefits}

  \begin{itemize}
  \item \textbf{less} things
    \pp
  \item we use package dependency resolution mechanism
    \pp
  \item we let rpm/deb/pacman handle 90\% of the installation
    \pp
  \item we don't pull files from the host
    \pp
  \item images can be reproducible
    \pp
  \item images are the same for everyone
    \pp
  \item images can be easily signed
    \pp
  \item systemd does the heavy lifting in the initrd
    \pp
  \item bash helpers → compiled programs
    \pp
  \item developers don't need to learn another system\\
    (initrd is like a normal system,\\
    \phantom( just on an fs not backed by a disk)
    \pp
  \item clear ownership of bugs
    \pp
  \item initrd infrastructure can be shared between distros
  \end{itemize}
\end{frame}

\begin{frame}
  \frametitle{Boot loader integration}

  Credentials and sysexts are loaded by systemd-stub.\\
  Loading \texttt{should} work with any boot loader.

  % \quad
\end{frame}

\begin{frame}[fragile]
  \frametitle{Links}

  \url{https://github.com/systemd/mkosi}

  \url{https://github.com/systemd/mkosi-initrd}

  \url{https://www.freedesktop.org/software/systemd/man/systemd-sysext.html}

  {
    \small
    \url{https://gitlab.com/cryptsetup/cryptsetup/-/wikis/DMVerity}\\
    \color{gray}{\url{https://www.kernel.org/doc/html/latest/admin-guide/device-mapper/verity.html}}
    }

  \url{https://www.kernel.org/doc/html/latest/filesystems/overlayfs.html}

  \quad

  These slides:
  \url{https://github.com/keszybz/mkosi-initrd-talk/raw/main/lpc2022-new-design-for-initrds.pdf}

  \quad
  \pp

  \hfill \textcolor{red}{QUESTIONS?} \textcolor{green}{\bf /} \textcolor{red}{EOF} \hfill{}

\end{frame}


\end{document}
