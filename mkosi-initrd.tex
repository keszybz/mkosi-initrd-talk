\documentclass[]{beamer}
%\usepackage[utopia]{mathdesign}
%\usepackage[no-math]{fontspec}
%\setmainfont{Liberation Serif}

\usepackage{minted}
% \usepackage{redhat}
\usepackage{hyperref}
\usepackage{ccicons}
\usepackage{ulem}

\usepackage{tikz}
\usetikzlibrary{arrows,shapes,snakes,automata,backgrounds,petri}

%gets rid of bottom navigation bars
\setbeamertemplate{footline}[page number]{}

%gets rid of navigation symbols
\setbeamertemplate{navigation symbols}{}

\newcommand{\done}{\textcolor{teal}{\checkmark}}
\newcommand{\pull}[3]{\href{https://github.com/systemd/#1/pulls/#2}{#3 (\##2)}}
\newcommand{\pulldone}[3]{\pull{#1}{#2}{#3} \done}
\newcommand{\commit}[3]{\href{https://github.com/systemd/#1/commit/#2}{#3 (#2)}}
\newcommand{\commitdone}[3]{\commit{#1}{#2}{#3} \done}

\title{Building initrd images from rpms}
\author{Zbigniew Jędrzejewski-Szmek}
\institute{%
  \includegraphics[width=0.4\textwidth]{images/Logo-redhat-color-375.png}\\
  \medskip
  \textit{zbyszek@in.waw.pl}\\
  \medskip
  \ccbysa
}
\date{\tiny DevConf2022, 28.01.2022}

\begin{document}
\begin{frame}
\titlepage % Print the title page as the first slide
\end{frame}

\begin{frame}
  \frametitle{Why do we need an initrd?}

  a small file system that the boot loader passes to the kernel\\

  \pause

  the purpose of the initrd is to mount the real root file system\\

  \pause

  initrd == initramfs
\end{frame}

% ArchLinux: mkinitcpio
% https://wiki.archlinux.org/title/Mkinitcpio
% https://man.archlinux.org/man/mkinitcpio.8
% glibc-based, busybox, systemd optional?, runtime hooks
% Helpers: add_module, add_binary, add_file, add_dir, add_full_dir,
% add_symlink, add_all_modules, add_checked_modules, add_runscripts

% Debian: initramfs-tools, update-initramfs → mkinitramfs
% hooks, prereqs, copy_exec, force_load, log_*

\begin{frame}
  \frametitle{Status quo — dracut}

  Dracut — ``generic initramfs intrastructure''

  \begin{itemize}
  \item configuration mechanism for deciding what is available in the initrd image
    (also a dependency mechanism with \texttt{check()}, \texttt{depends()})
  \item create the image from files on the host
    (\texttt{instmods()}, \texttt{dracut\_install()}, \texttt{inst()},
    \texttt{inst\_hook()}, \texttt{inst\_rules()})
  \item event-driven execution queue in initrd
  \item helpers to do various things in the initrd
  \end{itemize}

%%  \bigskip

%%   \hrule

%%   \bigskip
%%   \pause

%%   Dracut (optionally) installs systemd and udev into the initrd.\\
%%   \pause
%%   Systemd is written to support running in the initrd.\\
%%   Systemd provides services to do all the things that the initrd does.\\

%%   \pause

%%   Why wrap systemd in another execution queue?
\end{frame}

\begin{frame}
  \frametitle{``initrds over the years''}

  \pause


  stage I: ``very special'' — busybox with scripts, ...

  \pause

  stage II: ``special'' — normal programs + custom event queue

  \pause

  stage III: ``quasinormal'' – normal programs + systemd + custom elements

  \pause

  stage IV: ``normal`` — just systemd + normal services
\end{frame}

\begin{frame}
  \frametitle{Xorg --> Wayland, ca. 2008}

  technical debt

  same people

  long-term coexistence
\end{frame}

\begin{frame}
  \frametitle{Why is the initrd so special?}

  As far as the kernel is concerned, the initrd is just another file system.\\

  The user-space is a bit different: 
  \texttt{/init}, \texttt{/etc/initrd-release}\\

%%   \pause
%%   (We switch back into the initrd image for shutdown.)\\

  \pause

  Anything we would do in the initrd, we also need to do in the host:\\
  storage\\
  degraded storage\\
  networking\\
  network-based file systems and storage (nfs, iscsi, clevis)\\
  fsck\\
  emergency mode\\

  \pause

  Nowadays all this functionality is implemented either using daemons
  and/or systemd units and/or various helpers.
\end{frame}

\begin{frame}
  \frametitle{What happens when a new file, a service, is added to a package?}

  packaging + dracut packaging
\end{frame}

\begin{frame}
  \frametitle{Overview of the new scheme}

  \begin{columns}
    \begin{column}[t]{0.5\textwidth}
      Dracut:
      \begin{itemize}
      \item configuration mechanism…
      \item create the image from files on the host
      \item event-driven execution queue…
      \item helpers … in the initrd
      \end{itemize}
    \end{column}         % Note: can't really replace just part of this, it's all tied together
    \pause

    \begin{column}[t]{0.5\textwidth}
      New scheme:
      \begin{itemize}
      \item a list of rpms
        \pause
      \item \texttt{dnf --installroot=… \&\& cpio --create …}
        \pause
      \item \texttt{systemd}\\\phantom{X}
        \pause
      \item 
        \textcolor{gray}{ordinary daemons}
      \end{itemize}
    \end{column}
  \end{columns}
\end{frame}

%% \begin{frame}
%%   \frametitle{Some questions that could in principle be asked frequently}

%%   Q: My rpm is too large?!\\\pause
%%   A: Split it up!\\
%%   \phantom{A: }Container folks, cloud users, live cd creators, flatpak consumers,\\
%%   \phantom{A: }IoT engineers will all thank you.

%%   \bigskip
%%   \pause

%%   Q: My rpm has too many dependencies?!\\
%%   A: … \\\phantom{A: }
%%      \url{https://docs.fedoraproject.org/en-US/minimization/}

%%   \bigskip
%%   \pause

%%   Q: My rpm requires special setup / doesn't work in the initrd…\\
%%   A: …\\\phantom{A: }

%% \end{frame}

\begin{frame}
   \frametitle{Advantaged of building the image directly from packages}

  \begin{itemize}
  \item reliable installation: rpm is very good at doing what it does
  \item normal dependency mechanism
    \pause
  \item we don't pull files from the host
    \pause
  \item images are immutable
  \item images are reproducible
    \pause
  \item bash helpers → compiled programs
  \item developers don't need to learn another system
    \pause
  \item clear ownership of bugs
  \item any improvements are immediately shared
  \end{itemize}
\end{frame}

\begin{frame}
  \frametitle{\texttt{mkosi-initrd}}

  \begin{itemize}
    \item \href{https://github.com/systemd/mkosi}{mkosi} builds images from rpms

    \item \href{https://github.com/systemd/mkosi-initrd}{mkosi-initrd} uses mkosi to create a .cpio.zstd archive
  \end{itemize}

  \bigskip

  \hrule

  \bigskip
  \pause

  Some alternatives:
  \begin{itemize}
  \item osbuild
  \item kiwi-ng
  \end{itemize}
\end{frame}

\begin{frame}
  \frametitle{\texttt{Stages}}

  I. local generation (\textcolor{gray}{just like dracut})

  \pause
  II. generation in koji\\
  (a curated set of initrd variants + extensions)

  \pause
  III. signing by Fedora keys
  (with opt-in signature verification)
 \end{frame}



\begin{frame}[fragile]
  \frametitle{Generation}

  \texttt{sudo dnf install kernel-core}\\

  \texttt{kernel-install add <version> <image>}\\
  
  \texttt{mkosi -o initd.cpio.zstd --build-env=KERNEL\_VERSION=<version>}\\

%%   % First with summary
%%   \begin{minted}{bash}
%% sudo mkosi -o initrd.cpio.zstd summary
%% sudo mkosi -o initrd.cpio.zstd
%%   \end{minted}

%%   \bigskip
%%   \pause

%%   \begin{minted}{bash}
%% KVER=5.13.0-0.rc2.19.fc35.x86_64
%% sudo mkosi --build-env=KERNEL_VERSION=$KVER -f -o initrd-$KVER.cpio.zstd -r rawhide-35
%%   \end{minted}
\end{frame}

\begin{frame}[fragile]
  \frametitle{Size comparison}

  \begin{minted}{console}
$ du -sh dracut-*.cpio.* mkosi-*.cpio.*
34M     dracut-5.13.4-200.fc34.x86_64.cpio.xz
62M     mkosi-5.13.4-200.fc34.x86_64.cpio.zstd
  \end{minted}
  \pause
  \begin{minted}{console}
$ du -sh dracut-*.d/ mkosi-*.d/
77M     dracut-5.13.4-200.fc34.x86_64.d
165M    mkosi-5.13.4-200.fc34.x86_64.d
  \end{minted}

  \pause
  \bigskip

  Some differences:\\
  /lib/modules  5 MB vs. 37 MB\\
  /usr/bin      8 MB vs. 18 MB\\
  /usr/sbin    10 MB vs. 14 MB\\
  /usr/lib64   41 MB vs. 51 MB\\
  /usr/share    0.5 MB vs. 11 MB\\
  (…/licenses 3 MB, …/zoneinfo 5 MB, …/pki 1 MB, …/terminfo 1 MB)\\
  /etc          0.5 MB vs. 12 MB\\
  (…/udev/hwdb.bin 9MB, …/pki 1 MB)

\end{frame}

%% \begin{frame}
%%   \frametitle{Host-specific and Generic images}

%%   We build images on the host to be able to customize.\\
%%   Locally-built images are incompatible with centralized signing for Secure Boot.

%%   \bigskip
%%   \pause

%%   With images built from rpms it makes even less sense:\\
%%   everybody does the work, the result is always the same.

%%   \bigskip
%%   \pause

%%   We could easily build one huge image with ``everything'', but
%%   it would be slow and boot partitions are small.\\

%%   We need a mechanism to extend/customize images.
%% \end{frame}

\begin{frame}
  \frametitle{Extensions}

  systemd-sysext: extensions mounted with overlayfs\\

  \phantom{systemd-sysext: }dm-verity + signatures
\end{frame}

\begin{frame}
  \frametitle{Building sysexts with mkosi}

  \begin{enumerate}
  \item Mount an initramfs image somewhere
  \item Mount an OverlayFS over it (upper layer empty)
  \item \texttt{dnf install --installroot=… <packages for sysext>}
  \item Create a file system image with upper layer only
  \item (Optionally create partition dm-verity hash for it)
  \item (Optionally sign the whole thing)
  \end{enumerate}
\end{frame}

%% \begin{frame}
%%   \frametitle{\sout{Host-specific and }Generic and Host-customized images}

%%   We need a mechanism to extend/customize images.

%%   \bigskip
%%   \pause

%%   sysexts can be loaded into the initramfs image.

%%   \bigskip
%%   \pause

%%   Distro-side: the distro builds the kernel, and the initrd,\\
%%   \phantom{Distro-side: }and some set of sysexts\\
%%   \phantom{Distro-side: }(all three are signed)\\

%%   \pause

%%   The boot loader / firmware verifies the kernel+initrd combo\\
%%   The initrd checks and loads sysexts\\
%%   The kernel verifies sysext images using dm-verity\\
%%   (plenty of details TBD)

%%   % \vfill
%%   % \textcolor{gray}{Should \texttt{initrd-5.13.4-200.fc34.x86\_64.rpm} specify \texttt{Requires:systemd>=<version used>}?}
%% \end{frame}

\begin{frame}
  \frametitle{What works?}

  \textbf{OK:}\\
  creation of initrd\\
  integration with kernel scriptlets\\
  building and use of sysexts\\
  Fedora Server in QEmu with direct kernel boot\\
  Normal laptops, LVM, LUKS\\
  emergency mode without authentication\\
  resume

  \bigskip

  \textbf{Requires future work:}\\
  iscsi\\
  switching back to initramfs for shutdown\\
  firmware

  \bigskip

  \textbf{Not tested:}\\
  fcoe,
  nfs,
  nbd,
  kdump,
  network syntax (ip=/ifname=/rd.route=/…) supported by dracut and systemd-network-generator,
  plymouth,
  network,
  raid,
  sshd,
  bluetooth,
  netconsole
\end{frame}

%% \begin{frame}
%%   \frametitle{TODO list}

%%   mkosi: \pulldone{mkosi}{728}{cpio and zstd}\\
%%   \phantom{mkosi: }\pulldone{mkosi}{744}{RemoveFiles=}\\
%%   \phantom{mkosi: }\commitdone{mkosi}{v10}{Release}\\

%%   various systemd fixes (249+ should be OK)\\

%%   split modules out of \texttt{kernel-core.rpm}\\

%%   figure out how to deal with extensions\\
%%   figure out how to authenticate root for troubleshooting\\

%%   \texttt{kernel-install} plugin to call mkosi-initrd\\

%%   mkosi support of sysext image creation
%% \end{frame}

%% \begin{frame}
%%   \frametitle{TODO list: minimization}

%%   Requires:dbus→Recommends:dbus in systemd\\

%%   Prune from the initrd:\\
%%   \texttt{pcre} (we have \texttt{pcre2}),\\
%%   \texttt{libcap} (we have \texttt{libcap-ng}),\\
%%   \texttt{shadow-utils} (users are pre-created),\\
%%   \texttt{util-linux} (we have \texttt{util-linux-core}),\\
%%   \texttt{fedora-repos} (images are static),\\
%%   \texttt{alternatives} (wtf?),\\
%%   \texttt{tzdata}\\

%%   Port \texttt{systemd} over to \texttt{openssl}, drop \texttt{libgcrypt}, \texttt{libgpg-error}

%%   Drop \texttt{polkit}?

%%   Do something with identical license texts?
%% \end{frame}
 
\begin{frame}
  \frametitle{Summary}

  Build initramfs images directly from system packages\\
  Let systemd do the heavy lifting in the initrd\\
  Do things in the initrd like on the host\\
  Extend the initrd image using systemd-ext/OverlayFS\\
  (Build initrd images and extensions in koji)\\
  (Sign and verify all individual compoments)
\end{frame}

\begin{frame}[fragile]
  \frametitle{Links}

  \url{https://github.com/systemd/mkosi}

  \url{https://github.com/systemd/mkosi-initrd}

  \url{https://www.freedesktop.org/software/systemd/man/systemd-sysext.html}

  {
    \small
    \url{https://gitlab.com/cryptsetup/cryptsetup/-/wikis/DMVerity}\\
    \color{gray}{\url{https://www.kernel.org/doc/html/latest/admin-guide/device-mapper/verity.html}}
    }

  \url{https://www.kernel.org/doc/html/latest/filesystems/overlayfs.html}

  These slides: \url{https://github.com/keszybz/mkosi-initrd-talk}

\end{frame}

\end{document}
